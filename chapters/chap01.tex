\chapter{Introduction \& Problem Statement}
\label{chapter:intro}

In their paper "The unreasonable effectiveness of Data" \cite{norvig_eod} (alluding to "The unreasonable effectiveness of Mathematics in the Natural Sciences" by \cite{mathematics_unreasonable}) the authors posed one of the fundamental axioms in machine learning today, (often even invoked as shorthand) the fact that a larger set of training data would \textbf{always} be preferable, all other things being equal.

Ultimately, predictive power of machine learning models seems to be driven by

\begin{itemize}
	\item Size and quality of training data
	\item Expressive power of models (itself limited by processing capabilities)
\end{itemize}

That is, assuming that a model which fits the task at hand exists and has sufficient power to learn, the deciding factor in successful application of machine learning becomes the quality and amount of data.

The number of books (such as the one by \cite{dataprep}) on the topic of data preparation - that is, reformulation of existing data to facilitate the extraction of patterns from it -  gives some indication on the importance of this step. 

\pagebreak

The example given by \cite{geron2019hands} in his book of predicting time of day from the image of a clock face illustrates this most clearly: while it would be possible to train a neural network on a large number of images of clock faces, forcing them to infer the times the hands of the clock point at eventually - by simply reformulating the image into a polar coordinate system of hands pointing at certain angles, however, the problem may be solved directly with a few lines of code.

\cite{datawrangling_time} showed in their survey on the importance of different steps in the process of creating a predictive model, that the preparation of data occupies a significant share of time spent by data scientists attempting to apply machine learning methodologies.

Of significance is the implied notion that while data preparation is a comparatively time-intensive part of the process of machine learning, this is the case since effort spent in data preparation is a lever to increase model performance through more data scientist work. Not all steps of the process offer the same (potential) tradeoff between work and predictive power. In the same paper \cite{datawrangling_time} notes that model tuning only consumes on the order of 14\% of data scientist's time. This allocation of time follows from the assumption that it is \textbf{more likely} that the ultimate degree of success on a task will be affected by enriching the data at hand than it is to innovate an entirely new method of prediction, data clustering or inference.

Furthermore, while most books on the topic mention the acquisition of additional data and the benefits of enriching and reformulating the data at hand, it is almost universally understood that the objective is making ideal use of the data at hand.

The amount of data that is necessary to solve a given task can usually not be defined, \cite{raudys1991small} attempt to quantify the effect of small sample size, on classification tasks specifically.

\pagebreak

Data size and model complexity are usually inversely correlated depending on the task. Scikit-learn, the de-facto standard library for python "shallow" machine learning (that is, not using deep \acp{nn}) offers a "decision map" on choosing the right model:

\fig{img/ml_map}{Decision Map for choosing the right estimator, from scikit learn \url{https://scikit-learn.org/stable/tutorial/machine_learning_map/index.html}}{fig:ml_map}{.8}

The takeaway here is the fact that many key "decision nodes" are based on the data size available.

The goal of this paper is to empirically test whether or not this strict relationship may be ameliorated by boosting the available training data using deep learning.

\pagebreak

\section{Objectives \& Methods}

In contrast to "shallow" learning methods, deep learning refers to creating intermediate representations in \acp{nn} via \textbf{stacked layers} at the cost of increased training data as \cite{swingler1996applying} explains.

In his excellent book on deep learning, keras author \cite{chollet2017deep} construes neural networks as structured hierarchical models which extract progressive representations from data, thereby enabling a more abstract understanding of patterns within the data.

The objective of this paper is therefore to test whether more complex deep learning models may be applied to a small dataset by synthetically enlarging the training data, effectively bypassing the need for extensive feature engineering by employing an outsized amount of computational power to a small dataset, effectively trading computation for data preparation time.

After a review of different approaches to creating synthetic data and their results in a later chapter (\ref{chapter:synthetic_data}), the practical viability of this approach will be tested empirically in chapter \ref{chapter:experiment_results}.

The method used to increase training data is a \ac{GAN} - to be introduced in a later chapter (\ref{chapter:gan}).
The \ac{GAN} will be developed on a well-known small machine learning dataset. After boosting the dataset size, different models will be trained on this dataset to compare their performances. If the results are promising, the method will be encapsulated into a python package and published on the \ac{pypi} - \url{https://pypi.org/}.
