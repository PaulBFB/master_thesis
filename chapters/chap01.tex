\chapter{Introduction}

In 2012 Krizhevsky and his colleagues entered and won the ImageNet classification contest with a deep convolutional neural network \ac{cnn}, outperforming other models by a significant margin \cite{krizhevsky2012imagenet}. 

Currently, data scientists spend a significant amount (how much? sources!) of their time, when solving 'shallow' machine learning tasks (such as???) in feature engineering / preprocessing. Source! examples!
This is due to the fact that shallow approaches such as decision trees, GBM and SVM models require features that 'directly' connect the prepared data to the searched-for outcome. (source) 

Deep learning (neural networks) eliminate the necessity of such preprocessing at the cost of increased training data (source) 

\blindtext \citet{Shearer2000}

\section{Problem Situation}

\blindtext

\fig{img/sax_approximated_series}{Sax approximation of a time series}{fig:sax}{0.5}

As can be seen in Figure~\ref{fig:sax} \ldots

\section{Objectives}

\blindtext

\section{Methods}

\blindtext

\section{Structure}

\blindtext

\section{Tables}

Table~\ref{tab:table-one} shows an example table.

\begin{table}[htbp]
    \centering
    \caption{This is a table}
    \label{tab:table-one}
    \begin{tabular}{lll}
        \addlinespace
        \toprule
        Column 1 & Column 2 & Column 3 \\
        \midrule
        A     & B     & C \\
        D     & E     & F \\
        G     & H     & I \\
        \bottomrule
    \end{tabular}
\end{table}

\section{Source Code}

\begin{lstlisting}[language=Java, caption=Hello World in Java, label=lst:hello-world-java]
public class Hello {
    public static void main(String[] args) {
        System.out.println("Hello World");
    }
}
\end{lstlisting}

Listing~\ref{lst:hello-world-java} shows the classic Hello World in Java.

\lstinputlisting[language=Python, caption=Hello World in Python, label=lst:hello-world-py]{./lst/hello.py}

Listing~\ref{lst:hello-world-py} shows the classic Hello World in Python.

\begin{lstlisting}[language=JavaScript, caption=Hello World in JavaScript, label=lst:hello-world-javascript]
function hello() {
    console.log('Hello World');
}

hello();
\end{lstlisting}

\newpage
\begin{lstlisting}[language=ES6, caption=Hello World in JavaScript (ES6), label=lst:hello-world-javascript]
const hello = async () => {
    await console.log('Hello World');
}

hello();
\end{lstlisting}

\section{Acronyms}
It is also possible to define abbreviations, such as \ac{html} and \ac{js}. We use the \texttt{acronym} packages that provides several options like plurals \acp{js} (see \url{ftp://ftp.tu-chemnitz.de/pub/tex/macros/latex/contrib/acronym/acronym.pdf}).
