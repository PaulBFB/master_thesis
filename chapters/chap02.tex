\chapter{Synthetic Data in Privacy}

%\blindtext[1] 
cite --> paper from source, different models on synthetic data!

\section{Synthetic Data for model performance}

%\Blindtext[4][1]
When training \acp{nn} for image classification, (source) a common practice is \textbf{data augmentation}, a range of random transformation applied to images in order to synthetically increase the breadth of data that the model is exposed to. 
Such operations include 
\begin{itemize}
	\item rotation
	\item shearing
	\item zoom
	\item height \& width shift
\end{itemize}

effectively, these operations transform an Image while preserving the underlying signals in the data. However, with other types of data this might be possible. Attributes of another dataset may not be feasibly 'shifted' in one direction or another without fundamentally changing the signal and misleading the model.

\textbf{note - the infeasibility of pretraining on non-image datasets - representations of the visual world}

\section{Deep Learning}

%\Blindtext[4][1]
